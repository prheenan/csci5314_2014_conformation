%%%%%%%%%%%%%%%%%%%%%%%%%%%%%%%%%%%%%%%%%
% Short Sectioned Assignment
% LaTeX Template
% Version 1.0 (5/5/12)
%
% This template has been downloaded from:
% http://www.LaTeXTemplates.com
%
% Original author:
% Frits Wenneker (http://www.howtotex.com)
%
% License:
% CC BY-NC-SA 3.0 (http://creativecommons.org/licenses/by-nc-sa/3.0/)
%
%%%%%%%%%%%%%%%%%%%%%%%%%%%%%%%%%%%%%%%%%

%----------------------------------------------------------------------------------------
%	PACKAGES AND OTHER DOCUMENT CONFIGURATIONS
%----------------------------------------------------------------------------------------

\documentclass[paper=a4, fontsize=11pt]{scrartcl} % A4 paper and 11pt font size

\usepackage[T1]{fontenc} % Use 8-bit encoding that has 256 glyphs
\usepackage[english]{babel} % English language/hyphenation
\usepackage{amsmath,amsfonts,amsthm} % Math packages



\usepackage{fancyhdr} % Custom headers and footers
\pagestyle{fancyplain} % Makes all pages in the document conform to the custom headers and footers
\fancyhead{} % No page header - if you want one, create it in the same way as the footers below
\fancyfoot[L]{} % Empty left footer
\fancyfoot[C]{} % Empty center footer
\fancyfoot[R]{\thepage} % Page numbering for right footer
\renewcommand{\headrulewidth}{0pt} % Remove header underlines
\renewcommand{\footrulewidth}{0pt} % Remove footer underlines
\setlength{\headheight}{13.6pt} % Customize the height of the header

\numberwithin{equation}{section} % Number equations within sections (i.e. 1.1, 1.2, 2.1, 2.2 instead of 1, 2, 3, 4)
\numberwithin{figure}{section} % Number figures within sections (i.e. 1.1, 1.2, 2.1, 2.2 instead of 1, 2, 3, 4)
\numberwithin{table}{section} % Number tables within sections (i.e. 1.1, 1.2, 2.1, 2.2 instead of 1, 2, 3, 4)

\setlength\parindent{0pt} % Removes all indentation from paragraphs - comment this line for an assignment with lots of text

%----------------------------------------------------------------------------------------
%	TITLE SECTION
%----------------------------------------------------------------------------------------

\newcommand{\horrule}[1]{\rule{\linewidth}{#1}} % Create horizontal rule command with 1 argument of height
\newcommand{\ndiffn}[3]{\frac{\partial^{#3}#1}{\partial^{#3}#2}}
\newcommand{\diffn}[2]{\ndiffn{#1}{#2}{}}
\newcommand{\vect}[1]{\bold{#1}}

\newcommand{\eqs}[1]{
\begin{align} 
\begin{split}
#1
\end{split}					
\end{align}}


\title{	
\normalfont \normalsize 
\textsc{The University of Colorado at Boulder} \\ [25pt] % Your university, school and/or department name(s)
\horrule{0.5pt} \\[0.4cm] % Thin top horizontal rule
%\huge Assignment Title \\ % The assignment title
\horrule{2pt} \\[0.5cm] % Thick bottom horizontal rule
}

\author{Patrick Heenan} % Your name

\date{\normalsize\today} % Today's date or a custom date

\begin{document}

\maketitle % Print the title

%----------------------------------------------------------------------------------------
%	PROBLEM 1
%----------------------------------------------------------------------------------------

\section{Diffusion Coefficient in two dimensions}

In two dimensions, the density 'u' of particles freely diffusing in the x-y Cartesian plan of a system of particles is given by:

\eqs{
\diffn{u}{t} &=  D \nabla u\\ 
&= D [\ndiffn{u}{x}{2} + \ndiffn{u}{x}{2}] \\
}

Where D is the diffusion coefficient, with units of [m$^2$]/s. Assuming we know the initial condition of the particles exactly, we can find write the initial location of the particle using $\delta(x)$, which is 1 if x is 0 and 0 otherwise:

\eqs{ u(t_0) = \delta(x-x_0) \times \delta(y-y_0) }

Where we implicitly assume the starting density is normalized to 1. Using separation by variables, we have:

\eqs{ u(x,y,t) = F(x)G(y)H(t)}

\eqs{
\diffn{u}{t} = F\times G \times \diffn{H}{t}
&= D [G \times H \times \ndiffn{F}{x}{2} + F \times H \times \ndiffn{G}{Y}{2}] }

Dividing through by u, we have

\eqs{ \frac{1}{D \times H} \diffn{H}{t}
&= [ \frac{1}{F} \ndiffn{F}{x}{2} + \frac{1}{G} \ndiffn{G}{Y}{2}] }

Both of these sides are equal to some constant, since the left is only dependent on time, and the right is only dependent on x and y:


\eqs{
 \diffn{H}{t} &= -\lambda \times D \times H(t) \\ }

Similarly for F and G: 

\eqs{ \frac{1}{F} \ndiffn{F}{x}{2}  = -(\frac{1}{G} \ndiffn{G}{Y}{2} + \lambda)
  =-\gamma }


So that 

\eqs{
 \diffn{H}{t} &= -\lambda \times D \times H(t) \\ 
 \ndiffn{F}{x}{2} &= - \gamma \times F  \\
\ndiffn{G}{y}{2} &=  -[\lambda - \gamma] \times G  = - \beta \times B}

This can be solved by the normal methods:

\eqs{
H(t) &= A_h \times e^{-\lambda D \times t} \\
F(x) &= A_{f,1} \times e^{-\lambda x} + A_{f,2} \times e^{\lambda x} \\
G(y) &= A_{f,1} \times e^{-\beta y} + A_{f,2} \times e^{\beta x} \\}

Given that x and y are known, we have at u($t_0$) the solution to this system of equations is (Chandrashekhar 1943, equation 172):

\eqs{
 u(x,y,t) 
&= \frac{1}{(2 \pi D t)^{3/2}} e^{-\frac{[X - X_0]^2[Y-Y_0]^2}{4 D t}  } \\
 &=  \frac{1}{(2 \pi D t)^{3/2}} e^{-\frac{|\vect{r} - \vect{r}_0|^2}{4 D t}  }  }

where $\vect{r} = x\hat{x} + y \hat{y}$. Note that this only works when the space for diffusion is much smaller than the characteristic diffusion length $\frac{L^2}{D}$. The mean squared displacement is given by

\eqs{ MSD(t) 
  &= <(\vect{r} - \vect{r}_0)^2>_{avg} =  ||\vect{r}||^2 + ||\vect{r}_0||^2 - 2 \vect{r_0} <\vect{r}> \\
}

The appropriate integrals can be calculated:

\eqs{  <\vect{r}>
 &= \int_{r'=-\infty}^{r'=\infty} u(\vect{r'},t) \vect{r'} d\vect{r' } \\
&=  \int_{r'=-\infty}^{r'=\infty}
 \vect{r'} \frac{1}{(2 \pi D t)^{3/2}} e^{-\frac{|\vect{r'} - \vect{r}_0|^2}{4 D t}}  d\vect{r'} \\}

And similarly with $<(\vect{r})^2>$. Using the integral properties shown in (Chandrashekhar 1943, equation 172), we have 

\eqs{ \boxed{MSD = <(\vect{r} - \vect{r}_0)^2>_{avg} = 4 D t }}

For our purposes, we can calculate this value given the X and Y coordinates as:

\eqs{  <(\vect{r} - \vect{r}_0)^2>_{avg}\\
 &=  <(x_t - x_0)^2 + (y_t-y_0)^2>_{avg}  \\
&= \frac{1}{N-1} \sum_{t=1}^{N-1} [\Delta X^2 + \Delta Y^2] = 4Dt }








%----------------------------------------------------------------------------------------

\end{document}
